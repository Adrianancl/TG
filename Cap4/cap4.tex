\section{Code implementation}
\label{sec:codeimpl}

	\subsection{MATLAB environment}
		As mentioned on chapter \ref{ch:setup}, FLIR software provides indexed matrices in .mat format as output variables, which are MATLAB format of variables. Each pixel contains temperature information about itself, it is possible to visualize an example on a scaled image on the following figure:

		\begin{figure}[H]
			\centering
			\captionsetup{justification=centering}
			\includegraphics[scale=0.6]{Cap4/TempDist.jpg}
			\caption{Scaled image showing temperature distribution}
			\label{fig:tempdist}
		\end{figure}

		From figure \ref{fig:tempdist} with MATLAB Image Processing Toolbox support it possible to extract some informations about the image, such as:

		\begin{itemize}
			\item Edges recognition
			\item Image segmentation for tool, chip and workpiece
			\item Detection of tool tip
			\item Determine isotherms along tool
		\end{itemize}
	
	\subsection{Auxiliary functions}
		\subsubsection{Contour plot}
		\label{ch:seccontour}
			This is an important tool for this paper, contour plot is able to provide same level curves. Since the variable used on the process is a temperature matrix, this tool will calculate continuous lines, which the temperature of each pixel has very close value. Doing it with a small tolerance, the lines calculated are isotherms of the image. Then, with these lines it is also possible to extract its coordinates, which it will be essential to calculate heat carried away from volume control by means of tool.

			\begin{figure}[H]
				\centering
				\captionsetup{justification=centering}
				\includegraphics[scale=0.6]{Cap4/contour.jpg}
				\caption{Contour plot}
				\label{fig:contour}
			\end{figure}

		\subsubsection{Hough lines transformation}
		\label{ch:sechough}
			Hough transform is an extensive method used in computer vision. It is an extraction feature for complex geometries, using normal parameterization for straight lines \cite{duda1972use}. Concerning about the images, the rake and clearance face can be mapped by means of hough lines transformation in MATLAB. It is necessary to provide a probable range of angles in what the angular coefficient of the sought lines are defined. More precise is this range, more reliable and faster will be the output.

			The test bench, where the experiments were held, allows a fixed placement of tool.

			\begin{figure}[H]
				\centering
				\captionsetup{justification=centering}
				\includegraphics[scale = 0.65]{Cap4/imgset.jpg}
				\caption{Placement of tool}
				\label{fig:imgset}
			\end{figure}

			It means the angle between the rake face and horizontal line and the angle between clearance face and vertical line are always the designed rake and clearance angles, respectively. In other words, the tool does not rotate in relation to the reference axes. Because of this, it is possible to perform hough transformation on the image, being very accurate. As the rake and clearance angle are always $6^{o}$ and $3^{o}$, respectively, the hough transform processing will last a shorter time with predetermined angles than otherwise.

	\subsection{Implementation steps}
		\subsubsection{Overview}	

			The method of the program was able to identify the tool and chip shapes, then the analysis could extract and provide features that were essential for the results of this paper. By means of image processing and some input data, features like maximum cutting zone temperature, maximum chip temperature, heat flows through chip and tool are some examples of what the code is able to provide.	
			
		\subsubsection{Finding tool edges}

			As mentioned in the subsection \nameref{ch:sechough}, the method to find tool edges has to provide an accurate range of angles that the rake and clearance angles are inserted. The process is simple and it is demonstrated as follows:

			\lstinputlisting[firstline=223,lastline=257]{ApeA/TemperatureAnalyze.m}

			Since the rake angle is $6^{o}$ and the clearance is $3^{o}$ ranges of [81:85] and [2:5] were given to each respectively, as it is seen on lines 4 and 20. Regarding the rake angle, the range of angles is given by the complementary angles due to its reference in hough method. In this way, the hough transform returns highlighted points in the accumulation matrix of hough process and from them it is chosen the 10 first points to be analyzed, which is a reasonable amount of points that may represent sections of the edge lines.

			The fixed position of tool allows also the predetermination of the $\rho$ parameter, which is distance of the detected lines from the reference. This is also seen on lines 14 and 30 as boundary conditions to determine the right edge lines. The outputs of this function are the endings coordinates of the detected line and also the angle of the corresponding angular coefficient. 

		\subsubsection{Rake and clearance face}

			With the data provided by the output of hough function, it is possible to extend the lines to match the entire rake and clearance edge. This is an important step of the analysis method because it allows to build an object (binary image) that is a mask to remove only the region of interest, the tool shape in this case . Consequently, it will be possible analyze the temperature fields and thermal behavior inside the tool without any interference from the temperatures in the vicinity.

			\begin{figure}[H]
			\centering
			\captionsetup{justification=centering}
			\includegraphics[scale = 0.6]{Imagens/hough.png}
			\caption{Lines detected by hough transformation method}
			\label{fig:hough}
			\end{figure}

		\subsubsection{Tool tip coordinates}

			As the rake and clearance edges are determined, the tool tip will be calculated by means of the intersection between these lines. On the figure \ref{fig:hough} the found lines are extended until they intersect, then the coordinates of tool tip can be calculated. It is important to determine these coordinates due to the interest in knowing the temperatures of the area close to the tip and what is the maximum value it can reach, which is related directly with tool life and therefore the surface finish.

		\subsubsection{Maximum temperatures}

			As the code were able to segment the tool shape from the entire matrix, it gets easier to extract the other region of interest that present measurable range of temperatures, which is the chip. Getting the maximum temperature of each zone allows not only to know if the measured temperatures are inside the limit of measurement but also to compare the behavior of this maximum temperature of different cutting velocities and depths of cut.

		\subsubsection{Temperature fields}

			In this step, it will be used the auxiliary function mentioned on the subsection \nameref{ch:seccontour}. This is an important function to determine same level curves, as the isotherms inside the tool shape. The contour levels are determined in a step of 50 $^{o}$C.

		
			\lstinputlisting[firstline=383,lastline=383]{ApeA/TemperatureAnalyze.m}

			The output of contour function is a matrix C with 2 rows that will provide the levels of temperature and the number of coordinates followed by their absolute values of x and y, which are very valuable when comes to calculate heat flows.

			\begin{mdframed}[backgroundcolor=lightgray!25!]
			\begin{alltt}\fontsize{9pt}{8pt}\fontfamily{pcr}\selectfont
			C    =  [C(1) C(2) C(3) ...C(k)... C(N)]
			C(k) =  [level x(1) x(2)...
			         numxy y(1) y(2)...]
			\end{alltt}
			\end{mdframed}
			
			For each matrix C(k), level shows which temperature it is representing and numxy is the amount of coordinates used to build the level. The coordinates are represented in the pair (x,y).

		\subsubsection{Heat flows - Chip and Tool}
		\label{heatflows}
			As described on section \ref{methods}, the heat flow through tool and the energy carried away by chip are calculated. For heat flow through the tool, it is possible to extract isothermal lines by means of contour command and to calculate the gradient of temperatures  with gradient command, which already is normal to the isothermal lines due to its properties. The width is already known \ref{sec:exSetup}. The length of the chosen isotherm is done by counting the amount of pixels provided by the coordinates in contour plot and turned into millimeter with the scale factor afterwards.
			In the case of the energy carried away by chip, the chosen line is placed on the end of contact chip - tool. The explanation for it is that all the heat source in the friction zone is located before this line, in other words there is no other heat source after this line that could provide more thermal energy to be carried away by chip.		

		\subsubsection{Heat partitions}
			Having the results of the subsection \ref{methods}, these values can be combined with the total power ($P$) generated during the cutting process to calculate the energy that goes to the workpiece by means of energy balance (equation \ref{eq_energybalance}). Then, it is possible to calculate the heat partition relative to each zone of interest.

			\begin{equation} 
			\label{eq_heatpartition}
			p_{i} = \frac{\dot{Q}_{i}}{P}
			\end{equation}

			Which the index i is related to C (chip), W (workpiece) and T (tool).

	\section{Method validation}

		As described in the previous section \ref{sec:codeimpl}, there are many outputs of the implemented method, shear and normal stresses related to the mechanical part for example. However, in this paper the thermal modeling will be the focus of discussions.

		The total power produced along this high speed machining was calculated as in the equation \ref{eq_power}, the values are shown on figure \ref{fig:totPower}.

		\begin{figure}[H]
			\centering
			\captionsetup{justification=centering}
			\includegraphics[scale=0.55]{Imagens/Total_power.png}
			\caption{Total power produced}
			\label{fig:totPower}
		\end{figure}

		As expected, the higher are the values of cutting velocity or depth of cut higher are the values of total power produced.
		For each experiment, the computational method was able to provide the thermal energy that goes to tool, chip and workpiece by means of energy balance. Then, it can be observed the thermal behavior of every area of interest along the workpiece position. The measurement starts when a reasonable area of the cutting zone reaches the minimum measurable temperature. For cutting velocity of $150 m/min$ it starts earlier because the rate of heat production is higher than when the cutting velocity is $100 m/min$.

		\begin{figure}[H]
			\centering
			\captionsetup{justification=centering}
			\includegraphics[scale=0.55]{Imagens/Inner_Energy.png}
			\caption{Inner energy of tool along workpiece position}
			\label{fig:innerTool}
		\end{figure}

		\begin{figure}[H]
			\centering
			\captionsetup{justification=centering}
			\includegraphics[scale=0.55]{Imagens/energyTool2.png}
			\caption{Heat flow into tool}
			\label{fig:hflowTool}
		\end{figure}

		As it can be observed on the previous figure \ref{fig:hflowTool}, the change rate of the inner energy of tool begins with a higher value than in the end of process. The rate starts to stabilize, indicating the beginning of the steady state. 

		To exemplify the results, it will be taken to represent the outcomes regarding heat partitions the experiment with cutting velocity $v_{c} = 150 m/min$ and depth of cut $a_{p} = 500 \mu m$. All the others experiments had approximately the same behavior during the cutting process.
		Concerning the heat partition through tool, workpiece and the energy carried away by chip, their behaviors can be observed on the figure \ref{fig:hpartExp}. There is a slight decrement in the heat flow through tool, which it was expected due to the steady state as discussed before. As for the energy carried by chip, it may be noticed a slight increment.

		\begin{figure}[H]
			\centering
			\captionsetup{justification=centering}
			\includegraphics[scale=0.55]{Imagens/partition500150.png}
			\caption{Heat partition for experiment with a$_{p}$ = 500$\mu$m and v$_{c}$ = 150 m/min}
			\label{fig:hpartExp}
		\end{figure}

		It is important to highlight the total power produced during the cutting process, which has a significant value because of the high values of cutting velocity and force. Also, it must be noticed the amount of energy that goes to chip (figure \ref{fig:energyChip} and \ref{fig:hpartChip}). The chip takes around 70\% of the total energy produced, this fact may be explained due to the high temperatures that the region can reach and the high velocity of flowing.

		\begin{figure}[H]
			\centering
			\captionsetup{justification=centering}
			\includegraphics[scale=0.55]{Imagens/energyChip.png}
			\caption{Thermal energy into chip}
			\label{fig:energyChip}
		\end{figure}

		\begin{figure}[H]
			\centering
			\captionsetup{justification=centering}
			\includegraphics[scale=0.55]{Imagens/PartChip.png}
			\caption{Heat partition ratio for chip}
			\label{fig:hpartChip}
		\end{figure}

		As for the tool (figure \ref{fig:hpartTool}), the partition of energy reaches a much smaller range when close to the steady state. The values of the partition to tool in this stage goes from 4\% until 8\%.

		\begin{figure}[H]
			\centering
			\captionsetup{justification=centering}
			\includegraphics[scale=0.55]{Imagens/partTool.png}
			\caption{Heat partition ratio for tool}
			\label{fig:hpartTool}
		\end{figure}