	write


	% A manipula\c{c}{\~a}o de arquivos de geometria {\'e} usualmente associada ao uso de programas de CAD, que costumam ter licen\c{c}as caras devido {\`a} quantidade de recursos sofisticados oferecidos por eles. Esse gasto {\'e} facilmente justific{\'a}vel quando o trabalho envolve a aplica\c{c}{\~a}o de um n{\'u}mero grande dessas fun\c{c}{\~o}es, por exemplo, no desenvolvimento de uma nova geometria para um projeto. No entanto, para muitos usu{\'a}rios, a tarefa pode ser simples e requerer poucas fun\c{c}{\~o}es, por exemplo, para gerar um padr{\~a}o regular de proje\c{c}{\~o}es ao longo de uma superf{\'i}cie de modo a obter um conjunto de pontos para guiar a trajet{\'o}ria de uma ferramenta de usinagem durante o processo. Esses clientes teriam que gastar uma quantia elevada em licen\c{c}as, para ent{\~a}o usar apenas uma pequena por\c{c}{\~a}o do que o programa tem a oferecer.
	% Esse trabalho apresenta uma solu\c{c}{\~a}o que permite a leitura direta de um arquivo IGES e a gera\c{c}{\~a}o de um conjunto espec{\'i}fico de pontos usando fun\c{c}{\~o}es de c{\'o}digo aberto. Embora seja programado em $MATLAB^{TM}$, ele {\'e} compilado em um arquivo execut{\'a}vel, de forma que seja poss{\'i}vel seu uso sem a necessidade de licen\c{c}a.