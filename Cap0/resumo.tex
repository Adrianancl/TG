	M{\'e}todos de inspe\c{c}{\~a}o e monitoramento t{\^e}m sido utilizados cada vez mais para garantir a qualidade de processos. No campo da usinagem existem muitos par\^ametros importantes para assegurar que o processo forne\c{c}a os resultados estimados. O acabamento superficial de uma pe\c{c}a usinada e a vida \'util de uma ferramenta, por exemplo, sofrem influ\^encia direta da energia t\'ermica gerada nas zonas de calor. Devido a isso, existem muitos m\'etodos te\'oricos para a modelagem de temperatura distribu\'ida pela zona de corte, mas ainda faltam ferramentas que possam permitir a valida\c{c}{\~a}o pr\'atica de tais m\'etodos. Embora ainda existam desafios no uso adequado da termografia, essa tecnologia faz poss\'ivel o desenvolvimento de m\'etodos computacionais para o processamento de imagens t\'ermicas e, consequentemente a posterior an\'alise de fluxos de calor e parti\c{c}{\~o}es dessa energia.
	Este trabalho apresenta um m\'etodo computacional desenvolvido em MATLAB, com o suporte da toolbox de processamento de imagens, para an\'alise de imagens t\'ermicas, fornecendo resultados de campos de temperatura, energias internas, fluxos de calor e outras vari\'aveis de interesse que possam ser utilizadas no monitoramento da usinagem e no estudos de melhores par\^ametros de corte.