The computational method had a simple implementation, requiring the support of the image processing toolbox of MATLAB. The combination of a high speed camera with the fast processing of thermal images provided a tool for fast inspection of orthogonal cutting. The results were obtained for AISI 1045 steel, but it can be easily adapted for other types of materials by providing the new cutting parameters inside the source code.

The results found when processing thermal images provided a reasonable understanding about heat distribution throughout tool and chip components. Most part of the data provided by the cutting process regards to transient state, but is also possible to note it reaching the steady state close to the end of the cutting process, which this computational method is also able to analyze for this part of the process. It can be a valuable tool for validation of future studies concerning heat sources modeling and simulation.

Most of the heat generated during the cutting process is removed by the chip, about 70\% of the total generated power. The steady state showed a partition for the tool of about 6\%, which agrees with the model proposed by \citeonline{trigger1942}. The chip and the workpiece have bigger heat partitions due to their high flowing velocity, while the velocity of heat conduction inside the tool is much smaller.

One of the problems to elaborate this work was that many of the videos were damaged due to chip obstruction interfering on the ideal visualization of each thermal frame. Pieces of chip with different temperatures were captured on tool surface, disturbing the field of temperatures along the tool shape. This fact made it impossible to use some frames from the same video and sometimes entire experiments.

The thermography method for temperature measurement still presents some challenges, mainly when it comes to setting the correct emissivity. Even when coating the tool and the workpiece with black ink and conductiong experiments to determine the its emissivity, the ink cracks close to the tool tip and along the chip. This fact can be a source of error providing an overestimation of the emissivity value and consequently an underestimation of the real temperature. But even taking a reasonable effort to determine the right emissivity for accomplishing a reliable measurement, the termography is still a powerful tool for inspection, specially for cutting processes as discussed in this paper. With a camera filter capable of measure temperatures lower than 200 Celsius degrees, it would be possible to complete the study with the measurement of temperatures on the workpiece area, providing more results.

Computer vision, as well as image recognition patterns and image processing, is being used each time more in nowadays processes. For a future study beyond the scope of this paper, computer vision can become an even stronger tool when combined with machine learning, which is revolutionizing the most diverse areas. The principles used to build this computational method could be converted to analyze others types of cutting processes, such as milling. Then, it could be turned into an intelligent system to support machining processes, improving all cutting parameters in order to obtain higher tool efficiency, increasing tool life, improving surface finishing of the workpiece and reducing cutting time. 

Finally, it is also important to highlight that the method can be developed in other programming languages, such as Python and C++. Although the FLIR software has a direct connection with MATLAB, some programming languages do not require a paid lincense, which can be very interesting for making the solution cheaper.
