	For each experiment, the computational method was able to provide the thermal enegy that goes to tool, chip and, by means of energy balance, workpiece. Then, it can be observed the thermal behavior of every area of interest along the cutting time.
	
	% \begin{table}
	% \centering
	% \captionsetup{justification=centering}
	% \begin{tabular}{c c c c c}
	% \hline
	% \multirow{2}{*}{Tolerance} & \multicolumn{4}{c}{Number of iterations}\\
	%  & 1 & 2-5 & 6-10 & 11-20\\\hline
	% $10^{-1}$ & 7421 & 0 & 0 & 0\\
	% $10^{-2}$ & 7397 & 24 & 0 & 0\\
	% $10^{-3}$ & 7389 & 8 & 24 & 0\\
	% $10^{-4}$ & 7381 & 0 & 24 & 16\\
	% $10^{-5}$ & 7373 & 0 & 0 & 48\\
	% $10^{-6}$ & 7331 & 0 & 0 & 90\\
	% \hline
	% \end{tabular}
	% \caption{Number of points demanding each number of iterations for different tolerances. Grid size: $181 \times 41$.}
	% \label{tab:projpt_numit}
	% \end{table}