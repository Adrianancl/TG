	For each experiment, the computational method was able to provide the thermal enegy that goes to tool, chip and, by means of energy balance, workpiece. Then, it can be observed the thermal behavior of every area of interest along the cutting time.
	To exemplify the results, it will be taken to represent the outcomes concerning heat flows and heat partitions the experiment with cutting velocity $v_{c} = 150 m/min$ and depth of cut $a_{p} = 500 \mu m$. All the others experiments had approximately the same behavior along time.

	gráfico INNER ENERGY TOOL

	FIGURA variation INNER ENERGY TOOL

	As it can be observed on the previous figure REFERENCIA, the change rate of the inner energy of tool begins with a higher value than in the end of process. The rate starts to stabilize, indicating the beginnig of the steady state. 

	Concerning about the heat flow through tool, workpiece and the energy carried away by chip, their behaviors can be observed on the figure below REFERENCIA. There is a slight decrement in the heat flow through tool, which it was expected due to the steady state as discussed before. As for the energy carried by chip, it may be noticed a slight increment.

	FIGURA DE heat flows

	It is important to highlight the total power produced during the cutting process, which has a significant value because of the high values of cutting velocity and force.  the amount of energy that goes to the flowing chip.





	
	% \begin{table}
	% \centering
	% \captionsetup{justification=centering}
	% \begin{tabular}{c c c c c}
	% \hline
	% \multirow{2}{*}{Tolerance} & \multicolumn{4}{c}{Number of iterations}\\
	%  & 1 & 2-5 & 6-10 & 11-20\\\hline
	% $10^{-1}$ & 7421 & 0 & 0 & 0\\
	% $10^{-2}$ & 7397 & 24 & 0 & 0\\
	% $10^{-3}$ & 7389 & 8 & 24 & 0\\
	% $10^{-4}$ & 7381 & 0 & 24 & 16\\
	% $10^{-5}$ & 7373 & 0 & 0 & 48\\
	% $10^{-6}$ & 7331 & 0 & 0 & 90\\
	% \hline
	% \end{tabular}
	% \caption{Number of points demanding each number of iterations for different tolerances. Grid size: $181 \times 41$.}
	% \label{tab:projpt_numit}
	% \end{table}