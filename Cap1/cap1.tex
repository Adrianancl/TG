	In many machining cases, orthogonal cutting may be considered a good approximation to perform on the major cutting edge, that is why it has been extensively studied \cite{shaw2005metal}. For instance, planing and facing processes are some examples that orthogonal cutting conditions can be observed.

	Also, it is known that thermal behavior during the cutting process, as temperature fields and heat flows, has an important influence on tool life, surface finish and metallurgical structure of workpiece and machinability. Then, the study of thermal analysis on orthogonal cutting case shall be able to provide a better comprehension of many studies concerning thermal modeling of metal cutting processes \cite{komanduri2000thermal}, \cite{komanduri2001thermal}.
	
	\section{Overview of metal cutting}
	
	There are different ways to change raw material, as additive and subtractive \cite{shaw2005metal}. The additive processes occur when separated materials are put together, like 3D printing. On the other hand, subtractive way removes unnecessary material, which happens for machining processes as turning, milling and, in this paper, orthogonal cutting. The cutting process is composed basically by chip, tool and workpiece (figure \ref{fig:heatZones}).
	Many parameters are responsible for the surface finishing of the workpiece, for example. Depth of cut, cutting velocity, cutting material are some of these parameters. It is fundamental to use the right parameters for each type of cutting process, otherwise it can harm the final result and the process itself.

	\section{Heat zones}

	In machining there are 3 main regions of interest from where comes the heat produced during the cutting process \cite{shaw2005metal}. The first area represented on figure \ref{fig:heatZones} is called primary shear zone and is located along the shear plane, which is the edge between the undeformed workpiece and the chip. The second area is the plane of contact between tool and chip, also known as secondary shear zone or friction zone. As for third area, it is related to the wear caused due the friction between tool and finished workpiece surface, due to it is called wear zone.

	\begin{figure}[h]
		\centering
		\captionsetup{justification=centering}
		\includegraphics[scale=0.5]{Imagens/heatZones.png}
		\caption{Regions of interest}
		\label{fig:heatZones}
	\end{figure}

	\section{Mechanics of orthogonal cutting}

	In this paper it will be show numerous correlations among forces, stresses and dimensions for example. For this purpose it is important to discuss geometrical correlations in the composite cutting force circle (figure \ref{fig:circlec}).

	\begin{figure}[h]
		\centering
		\captionsetup{justification=centering}
		\includegraphics[scale=0.5]{Cap1/circlec.png}
		\caption{Cutting forces \cite{shaw2005metal}}
		\label{fig:circlec}
	\end{figure}

	From the figure \ref{fig:circlec} it can be stated about forces on the primary shear zone reference $F_{S}$ and $N_{S}$:

	\begin{equation} 
	\label{}
	F_{S} = F_{P}\cos\phi - F_{Q}\sin\phi
	\end{equation}
	\begin{equation} 
	\label{}
	N_{S} = F_{Q}\cos\phi + F_{P}\sin\phi
	\end{equation}

	Also, for the forces on the chip flow reference:

	\begin{equation} 
	\label{}
	F_{C} = F_{P}\sin\alpha + F_{Q}\cos\alpha
	\end{equation}
	\begin{equation} 
	\label{}
	N_{C} = F_{P}\cos\alpha - F_{Q}\sin\alpha
	\end{equation}

	These equations provide all auxiliary forces related to the known passive force $F_{Q}$ and force on the cutting direction $F_{P}$. Now, the variables of interest can be easily calculated, as the friction coefficient:

	\begin{equation} 
	\label{eq_friction}
	\mu = \frac{F_{C}}{N_{C}} = \frac{F_{Q} + F_{P}\tan\alpha}{F_{P} - F_{Q}\tan\alpha}
	\end{equation}

	Now the equations concerning about stresses are:

	\begin{equation} 
	\label{}
	A_{S} = \frac{wa_{p}}{\sin\phi}
	\end{equation}

	\begin{equation} 
	\label{}
	\tau = \frac{F_{S}}{A_{S}} = \frac{(F_{P}\cos\phi - F_{Q}\sin\phi)\sin\phi}{wa_{p}}
	\end{equation}

	\begin{equation} 
	\label{}
	\sigma = \frac{N_{S}}{A_{S}} = \frac{(F_{P}\sin\phi + F_{Q}\cos\phi)\sin\phi}{wa_{p}}
	\end{equation}

	Where $A_{S}$ is the area of the shear plane, $\tau$ is the shear stress and $\sigma$ is the normal stress.

	Another important parameter is the cutting ratio $r$, which can provide an important relation between the main cutting velocity and the chip outlet velocity. It is found experimentally that there is no change in density of metal during the cutting process and also for $w/a_{p} \geq 5$ the width of the chip is the same of the workpiece. Then, the equations are:

	\begin{equation} 
	\label{}
	a_{p}wl = a_{pc}w_{c}l_{c}
	\end{equation}

	Where $a_{p}$, w and l are the depth of cut, width of cut and length of cut respectively. Then, the cutting ratio is defined by:

	\begin{equation} 
	\label{}
	r = \frac{a_{p}}{a_{pc}} = \frac{l_{c}}{l}
	\end{equation}

	Having the cutting ratio, it is now possible to correlate cutting velocity $v$ and chip outlet velocity $v_{c}$ by means of the following equation:

	\begin{equation} 
	\label{}
	v_{c} = rv
	\end{equation}
	
	\section{Objective}

	The aim of this paper is to develop a computational method to analyze thermal images generated during the orthogonal cutting of AISI 1045 metal, focusing on the transient state due to the short time of cutting. It will be analyzed temperature distribution along the cutting tool, heat flows through tool, chip and workpiece.

	\section{Structure}
	
	This work is divided into 6 Chapters, including this \textbf{\nameref{ch:intro}}, plus one Appendix.
	
	The second chapter, \textbf{\nameref{ch:state}}, describes the existing technology which is relevant for the scope of this paper and upon which the work was built.

	The third chapter, \textbf{\nameref{ch:setup}}, describes the methodology and materials that conducted the experiments.
	
	The fourth, \textbf{\nameref{ch:implementation}}, describes the logical implementation of the final analysis code.
	
	The fifth, \textbf{\nameref{ch:results}}, presents the results and discussions about the outputs of the final code.
	
	The sixth and final chapter, \textbf{\nameref{ch:conclusions}}, sums up what was accomplished in this work and suggests how it may be expanded for new processes.
	
	The Appendix~\textbf{\nameref{ch:code}} contains all the code written for the program.