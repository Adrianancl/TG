	In many machining cases, orthogonal cutting may be considered a good approximation to performance on the major cutting edge, that is why it has been extensively studied \cite {shaw2005metal}. For instance, planing and facing processes are some examples that orthogonal cutting conditions can be observed.

	Also, it is knonw that thermal behavior during the cutting process, as temperature fields and heat flows, has a important influence on tool life, surface finish and mattalurgical structure of workpiece and machinability. Then, the study of thermal analysis on orthogonal cutting case shall be able to provide a better comprehension of many studies concerning thermal modeling of metal cuting processes \cite {komanduri2000thermal}, \cite{komanduri2001thermal}.
	
	The aim of this paper is to develop a method to analyze thermal images generated during the orthogonal cutting of AISI 1045 metal, focusing on the trasient state due to the short time of cutting. It will be analyzed temperature distribuition along the cutting tool, heat flows through tool, chip and workpiece.
	
	\section{Structure}
	
	This work is divided into 6 Chapters, including this \textbf{\nameref{ch:intro}}, plus one Appendix.
	
	The second chapter, \textbf{\nameref{ch:state}}, describes the existing technology which is relevant for the scope of this paper and upon which the work was built.

	The third chapter, \textbf{\nameref{ch:setup}}, describes the methodology and materials that conducted the experiments.
	
	The fourth, \textbf{\nameref{ch:implementation}}, describes the logical implementation of the final analysis code.
	
	The fifth, \textbf{\nameref{ch:results}}, presents the results and dicussions about the outputs of the final code.
	
	The sixth and final chapter, \textbf{\nameref{ch:conclusions}}, sums up what was accomplished in this work and sugests how it may be expanded for new processes.
	
	The Appendix~\textbf{\nameref{ch:code}} contains all the code written for the program.