	In many machining cases, orthogonal cutting may be considered a good approximation to perform on the major cutting edge, that is why it has been extensively studied \cite{shaw2005metal}. For instance, planing and facing processes are some examples that orthogonal cutting conditions can be observed.

	Also, it is known that thermal behavior during cutting processes, as temperature fields and heat flows, has an important influence on tool life, surface finish and metallurgical structure of workpiece and machinability. Then, the study of thermal analysis on orthogonal cutting case shall be able to provide a better comprehension of many studies concerning thermal modeling of metal cutting \cite{komanduri2000thermal}, \cite{komanduri2001thermal}.
	
	\section{Overview of metal cutting}
	
	There are different ways to modify raw material, as additive and subtractive \cite{shaw2005metal}. The additive processes occur when separated materials are put together, like 3D printing or welding. On the other hand, the subtractive way removes unnecessary material, which happens for machining processes as turning, milling and, in this paper, orthogonal cutting. The cutting process is composed basically by chip, tool and workpiece (figure \ref{fig:heatZones}).
	Many parameters are responsible for a good performance and final result, as surface finish of workpieces. Depth of cut, cutting velocity, cutting material are some of these parameters. It is fundamental to use the right parameters for each type of cutting process, otherwise it can damage the expected result and the process itself.
	
	\section{Objective}

	The aim of this paper is to develop a computational method to analyze thermal images generated during orthogonal cutting of AISI 1045 metal, focusing on the transient state due to the short time of cutting. It will be analyzed temperature distribution along the cutting tool, heat flows through tool, chip and workpiece.

	\section{Structure}
	
	This work is divided into 6 Chapters, including this \textbf{\nameref{ch:intro}}, plus one Appendix.
	
	The second chapter, \textbf{\nameref{ch:state}}, describes the existing technology which is relevant for the scope of this paper.

	The third chapter, \textbf{\nameref{ch:setup}}, describes the materials and methods that conducted the experiments.
	
	The fourth, \textbf{\nameref{ch:results}}, presents the results and discussions about code implementation and model validation.
	
	The fifth and final chapter, \textbf{\nameref{ch:conclusions}}, sums up what was accomplished in this work and suggests how it may be expanded for new processes.
	
	The Appendix~\textbf{\nameref{ch:code}} contains all the code written for the program.