	CAx is an acronym used for software which help a certain process, where "CA" stands for "Computer-Aided" and the "x" stands for a letter which characterizes the process. It is generally referred to as \textbf{Computer-Aided Technologies}.
	
	Most recent and advanced CAx tools merge many different stages of the product lifecycle management (PLM). The most common ones are Computer-Aided Design (CAD), Computer-Aided Engineering (CAE) and Computer-Aided Manufacturing (CAM).
	
	\subsection{Computer-aided manufacturing}
	
	The most common definition for Computer-aided manufacturing, or CAM, is the use of software to control machine tools and related ones in the manufacturing of workpieces. Its purpose is to increase the speed of the production process, as well as to increase dimensional precision and material consistency of components and tooling. In some cases, it can also minimize the waste of raw material, besides reducing energy consumption \cite{elanchezhian2007}.
	
	Although one of the ideas behind CAM is to avoid manual programming of numerical control (NC), it does not eliminate the need for skilled professionals such as manufacturing engineers, NC programmers and machinists. It actually leverages the value of skilled manufacturing professionals by providing advanced productivity tools, while also helping build the skills of new professionals by providing visualization, simulation and optimization tools, which are very useful for learning.
	
	The adoption of CAM software have recently risen due to the increasing ease of usage, with process wizards, templates and libraries, and due to increasing manufacturing complexity, which lead to the unfeasability of manual NC programming for certain operations, such as streamlining of tool paths, multi-function machining and multi-axis machining.
	
	Table~\ref{tab:listcam} lists the main CAM software companies, sorted by 2015 global revenue, with information extracted from the companies' websites.
	
	\begin{table}
	\centering
	\captionsetup{justification=centering}
	\begin{tabular}{l | l}
	\hline
	\textbf{Company} & \textbf{CAM software}\\\hline
	Dassault Syst{\`e}mes & CATIA\\
	Siemens & Solid Edge\\
	Vero Software & edgecam, work NC, surfcam\\
	AUTODESK & HSM, Powermill, Featurecam\\
	Geometric Technologies & CAMWorks\\
	OPEN MIND Technologies & HyperMill\\
	Tebis & Tebis\\
	CNC Software & Mastercam\\
	GibbsCAM 3D Systems & Cimatron\\
	PTC & Creo\\
	CG Tech & Vericut\\
	Missler Software & TopSolid\\
	Sprut Technology & Sprut Technology\\
	SAI Software & FlexiSign\\
	Gravotech Group & TYPE3\\
	MecSoft Corporation & MecSoft Corporation\\
	C\&G Systems & C\&G Systems\\
	SolidCAM & SolidCAM\\
	NTT Data Engineering Systems & NTT Data Engineerind Systems\\
	BobCAD-CAM & BobCAD-CAM\\
	\hline
	\end{tabular}
	\caption{Top 20 largest CAM companies by  global revenue in year 2015.}
	\label{tab:listcam}
	\end{table}
	
	\subsection{Rhinoceros 3D}
	
	Rhinoceros, also called Rhino or Rhino3D, is a commercial 3D computer graphics and CAD software created in 1980 by Robert McNeel \& Associates. Its geometry is based on the NURBS mathematical model, instead of polygon mesh-based applications. The NURBS model, explained in detail in Section~\ref{subsec:nurbs}, focuses on producing mathematically precise representation of curves and freeform surfaces.
	
	\begin{figure}[H]
		\centering
		\captionsetup{justification=centering}
		\includegraphics[width=0.4\linewidth]{Cap2/CAxSoftwares/Rhinoceros3D-logo.png}
		\caption{Rhinoceros 3D Logo, by Robert McNeel \& Associates.}
		\label{fig:Rhinoceros3D-logo}
	\end{figure}
	
	One main advantage Rhinoceros offers in comparison with other programs is that it supports Python as a scripting language. Through Python, one can use Rhinoscript, which is a library of Rhinoceros functions. Therefore, it allows for an interesting mixture of freedom to generate custom tool path patterns while at the same time offering tools to automatize this process, through Python's already existing libraries and functions.