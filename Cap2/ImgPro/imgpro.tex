Systems of vision have been very often approached with the current fast technology development and intelligent systems. They are used for the most diverse segments, as military and medical areas. Image processing has quickly gaining highlight. For instance, this is essential when comes to finding a pattern or extract an specific feature in a image. 

Colorful or gray scaled images can be treated as matrices with dimensions given by their pixel resolution. Each pixel corresponds to a cell inside this matrix and each cell contains a relevant information, it could be a level in grayscale, a coordinate or a temperature as in the case of this paper. Since they are matrices, they can be easily manipulated by means of mathematical operations and, consequently, processed to highlight one specific property or more.

Many studies have been and are being done about the use of image processing on machining. It was used to monitor flank wear of cutting tools \cite{jeon1988optical}, \cite{kurada1997machine} and also for detection of chatter during cutting processes \cite{khalifa2006image}. These are few examples of what image processing can do for machining industry, increasing tool life and surface finishing. There are uncountable ways which this tool can be applied to improve processes and quality of final products. The fast development of computer hardware makes the processing time of images continuously shorter, allowing systems of vision to be incorporated in online monitoring and then providing a real time feedback.