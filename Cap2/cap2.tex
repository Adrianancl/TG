\section{Thermal review}
	\subsection{Thermodynamics}

	\subsection{Heat zones in machining}

	In machining there are 3 main regions of interest from where comes the heat produced during the cutting process \cite{shaw2005metal}. The first area represented on figure \ref{fig:heatZones} is called primary shear zone and it is located along the shear plane, which is the boundary between undeformed workpiece and chip. The second area is the plane of contact between tool and chip, also known as secondary shear zone or friction zone. As for third one, it is related to the wear caused due the friction between tool and finished workpiece surface, due to it is called wear zone.

	\begin{figure}[h]
		\centering
		\captionsetup{justification=centering}
		\includegraphics[scale=0.5]{Imagens/heatZones.png}
		\caption{Regions of interest}
		\label{fig:heatZones}
	\end{figure}

	\subsection{Infrared thermography operation}

	Infrared termography is a non-contact way to measure infrared electromagnetic energy. The humam eye can not detect the range of infrared radiation. However, there are infrared cameras able to detect this energy and process the radiation into information (figure \ref{fig:thermog}).

	\begin{figure}[H]
		\centering
		\captionsetup{justification=centering}
		\includegraphics[scale=0.5]{Imagens/thermography.png}
		\caption{Radiation received by infrared camera \cite{usamentiaga2014}}
		\label{fig:thermog}
	\end{figure}

	It makes possible all thermal energy produced during a cutting process to be received by the infrared camera and to be synthesize in a matrix of temperatures afterwards. Since every body is able to emit infrared radiation when its temperature is above absolute zero, it is possible to observe contours of different bodies due to their temperature distribution. For this reason, thermography is a very important technology in military use, because it allows objects be seen even without proper illumination or in total lack of light situations.

	Thermography is able to work in two different ways: passive and active. The passive way occurs when the subject matter has its temperature different from the environment (often higher). On the other hand, the active way needs an external heat source to induce a reasonable contrast between the object and the background \cite{maldague2000}.

	As it can be observed on figure \ref{fig:thermog}, there are external sources of infrared radiation that can interfere on the target's temperature measurement. To correct the situation, the IR camera has a internal process called compensation \cite{usamentiaga2014}.

	The total energy received ($W_{tot}$) is composed by the sum of three parts, the emission from the main object ($E_{obj}$), the emission of the vicinity reflected by the object ($E_{refl}$) and the emission of the atmosphere ($E_{atm}$) as shown on figure \ref{fig:thermog}. Then it is possible to extract the real tempareture of the target object \cite{usamentiaga2014}.


\section{Mechanical review}
	\subsection{Mechanics of orthogonal cutting}

	In this section it will be shown numerous relations among forces, stresses and dimensions for example. For this purpose it is important to discuss geometrical correlations in the composite cutting force circle (figure \ref{fig:circlec}).

	\begin{figure}[h]
		\centering
		\captionsetup{justification=centering}
		\includegraphics[scale=0.5]{Cap1/circlec.png}
		\caption{Cutting forces \cite{shaw2005metal}}
		\label{fig:circlec}
	\end{figure}

	From the figure \ref{fig:circlec} it can be stated about forces on the primary shear zone reference $F_{S}$ and $N_{S}$:

	\begin{equation} 
	\label{}
	F_{S} = F_{P}\cos\phi - F_{Q}\sin\phi
	\end{equation}
	\begin{equation} 
	\label{}
	N_{S} = F_{Q}\cos\phi + F_{P}\sin\phi
	\end{equation}

	Also, for the forces on the chip flow direction reference:

	\begin{equation} 
	\label{}
	F_{C} = F_{P}\sin\alpha + F_{Q}\cos\alpha
	\end{equation}
	\begin{equation} 
	\label{}
	N_{C} = F_{P}\cos\alpha - F_{Q}\sin\alpha
	\end{equation}

	These equations provide all auxiliary forces related to the known passive force $F_{Q}$ and force on the cutting direction $F_{P}$. Now the variables of interest can be easily calculated, as the friction coefficient:

	\begin{equation} 
	\label{eq_friction}
	\mu = \frac{F_{C}}{N_{C}} = \frac{F_{Q} + F_{P}\tan\alpha}{F_{P} - F_{Q}\tan\alpha}
	\end{equation}

	Now the equations concerning about stresses are:

	\begin{equation} 
	\label{}
	A_{S} = \frac{wa_{p}}{\sin\phi}
	\end{equation}

	\begin{equation} 
	\label{}
	\tau = \frac{F_{S}}{A_{S}} = \frac{(F_{P}\cos\phi - F_{Q}\sin\phi)\sin\phi}{wa_{p}}
	\end{equation}

	\begin{equation} 
	\label{}
	\sigma = \frac{N_{S}}{A_{S}} = \frac{(F_{P}\sin\phi + F_{Q}\cos\phi)\sin\phi}{wa_{p}}
	\end{equation}

	Where $A_{S}$ is the area of the shear plane, $\tau$ is the shear stress and $\sigma$ is the normal stress.

	Another important parameter is the cutting ratio $r$, which can provide an important relation between the main cutting velocity and the chip outlet velocity. It is found experimentally that there is no change in density of metal during the cutting process and also when $w/a_{p} \geq 5$ makes the width of the chip the same of the workpiece. Then, the equations are:

	\begin{equation} 
	\label{}
	a_{p}wl = a_{pc}w_{c}l_{c}
	\end{equation}

	Where $a_{p}$, w and l are the depth of cut, width of cut and length of cut respectively. Then, the cutting ratio is defined by:

	\begin{equation} 
	\label{}
	r = \frac{a_{p}}{a_{pc}} = \frac{l_{c}}{l}
	\end{equation}

	Having the cutting ratio, it is now possible to correlate cutting velocity $v$ and chip outlet velocity $v_{c}$ by means of the following equation:

	\begin{equation} 
	\label{}
	v_{c} = rv
	\end{equation}

\section{State of the Art}

\subsection{Infrared Termography}
	\label{sec:infrared}
		Infrared termography is a contactless way to measure infrared electromagnetic energy. It makes possible to observe contours of different bodies due to their temperature distribution, since every body is able to emit electromagnetic radiation when its temperature is above absolute zero. For this reason, it is a very important technology in military use, because it allows objects be seen even without proper illumination or in total lack of light situations.

	\begin{figure}[H]
		\centering
		\captionsetup{justification=centering}
		\includegraphics[scale=0.75]{Cap2/InfraRed/exinfrared.png}
		\caption{Infrared photography of a cutting process \cite{abukhshim2006heat}}
		\label{fig:exinfrared}
	\end{figure}

	The thermography tool is able to work in two different ways: passive and active. The passive way occurs when the subject matter has its temperature different from the environment (often higher). On the other hand, the active way needs an external heat source to induce a reasonable contrast between the object and the background \cite {maldague2000}.
	For the case under study, high speed thermography has its positive and negative points. On the positive side, it may be mentioned:
	\begin{itemize}
		\item Fast inspection rate (reasonable number of images of high speed cutting)
		\item Contactless (no interference during the cutting process)
		\item Easy interpretation of the results (indexed image with temperatures in each pixel)
	\end{itemize}
	But it is also important to mention the difficulties that in this method still prevail:
	\begin{itemize}
		\item Only a limited thickness can be measured (under the main surface)
		\item Determine a suitable emissivity is a chalenge (it changes with temperature variation)
	\end{itemize} %done

\subsection{Image Processing}
	Machine vision systems have often been approached with the current fast technology development and intelligent systems. They are used for the most diverse segments, such as the military and medical areas. Image processing has quickly gaining ground. For instance, this is essential when comes to finding a pattern or extract a specific feature in an image. 

Colored or gray scaled images can be treated as matrices with dimensions given by their pixel resolution. Each pixel corresponds to a cell inside this matrix and each cell contains a relevant information, which could be a level in grayscale, a coordinate or a temperature as in this paper. Since they are matrices, they can be easily manipulated by means of mathematical operations and consequently processed to highlight one specific property or more.

\begin{figure}[H]
	\centering
	\captionsetup{justification=centering}
	\includegraphics[scale=0.6]{Imagens/imgPro.png}
	\caption{Diagram of a machine vision system \cite{sarma2009surface}}
	\label{fig:imgProcessing}
\end{figure}

There are many applications for image processing in the machining industry. \citeonline{sarma2009surface} developed a method for roughness determination ($R_{a}$), correlating gray scaled images with surface finish of glass fiber reinforced polymer (GFRP). After GFRP machining, images of the workpiece were taken by means of charge couple device camera and then processed (figure \ref{fig:imgProcessing}), obtaining a significant correlation between the predicted and real roughness.

\citeonline{jeon1988optical} and \citeonline{kurada1997machine} also developed an image processing method to monitor flank wear of cutting tools \emph{in situ}. Images in grayscale were taken and consequently processed for boundaries extraction, which indicates wear areas on tool tip surroundings.

Also, \citeonline{khalifa2006image} presented a method for chatter identification in turning processes, which is a significant challenge when comes to automatic machining processes. The vision system compares surface finish of workpieces machined under chatter and chatter-free conditions by means of roughness parameter. The method is also based on the behavior and distribution of gray levels in images of the workpiece.

 These are a few examples of what image processing can do for machining industry. There are uncountable other ways in which it can be applied to improve processes and quality of final products. The fast development of computer hardware makes the processing time of images continuously shorter, allowing vision systems to be incorporated in online monitoring and providing real time feedback.
