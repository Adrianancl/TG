	\section{Experimental Setup and Materials}
	\label{sec:exSetup}

		The experiments were carried out on Werkzeugmaschinenlabor (WZL) shop floor, located in Aachen in Germany, acquiring thermal images by means of a high speed infrared camera FLIR SC7600 (with frame rate of 328 fps and a resolution of 640 x 512 pixels), equipped with a macro lens 1:1 and FOV 9.6 x 7.7 mm. The test bench is set up that the tool stays in a fixed position in relation to the camera, keeping the relative distance between tool and camera constant. The scale factor provided by this setting was 15 $\mu$m/pixel. It allows for the metric conversion for future post processing of images.

		\begin{figure}[H]
			\centering
			\captionsetup{justification=centering}
			\includegraphics[scale = 0.5]{Cap3/exsetup.png}
			\caption{Experimental setup \cite{augspurger2016experimental}}
			\label{fig:exinfrared}
		\end{figure}

		An important factor for a reliable temperature measurement is the correct choice of the components' emissivity. To ease the emissivity determination, both the tool and the workpiece were coated with a black ink, allowing the emissivity evaluation for this case, which provided a value of $\epsilon = 0.85$. It is also important to highlight that the camera settings, factors as integration time and filters are also essential to determine a reliable measurements due to the amount of electromagnetic radiation received on the camera's sensors. The higher are the temperatures, higher is the energy produced and smaller should be the integration time, which is the time taken by the energy sensor to receive radiation and then convert it into temperature. The configurations were made to allow measurements in a range from 200 $^{o}$C to 900 $^{o}$C.
	
		The tool material was uncoated carbide insert (Sandvik H13A) with rake angle of 6$^{o}$, clearance angle of 3$^{o}$, cutting radius r$_{\beta}$ $\le$ 5$\mu$m and width 4.4 mm. The workpiece material was AISI 1045 normalized steel and its dimensions were 3.5 x 200 x 80 mm width, length and height, respectively. For the given range of temperature, the thermal conductivity was estimated to be $k = 75.4 W/mK$ and for tool heat capacity a regression function was built ($c(T)$) for corresponding temperature and heat capacity (equations \ref{eq_heatCapTool} and \ref{eq_heatCapWork}).
	
		For force acquisition during the process a three-component piezoelectric force platform was used, determining the cutting force and passive force. Since the cutting process was carried out in a linear and constant motion, it is possible to determine the overall power $P$ with velocity and cutting force. From the values obtained of forces along the cutting process, a mean value was taken to be used on power calculation, equation \ref{eq_power}. 
	
		All the experiments were held without coolant, with cutting speed of 100 $m/min$ and 150 $m/min$ and a$_{p}$ = [0.2, 0.3, 0.4, 0.5] mm (table \ref{tab:design}).

		The analysis method was built on MATLAB platform with the support of its image processing toolbox. FLIR software has a way of exporting the thermal images direct to .mat format, which are matrices projected to MATLAB environment. Each pixel from the exported images contains information about its position and temperature.

		\begin{table}[H]
			\centering
			\captionsetup{justification=centering}
			\includegraphics[scale = 0.6]{Cap3/tabexpset.png}
			\caption{Design of experiments \cite{augspurger2016experimental}}
			\label{tab:design}
		\end{table}

		As a machining process, the orthogonal cutting performance is subjected to many parameters like workpiece material, shape of tool, depth of cut and others. Because of it, the developed algorithm needs information about all these parameters to work as close as possible to real conditions. Then, all the necessary input data can be summarized on the following table:

		\begin{table}[H]
			\centering
			\captionsetup{justification=centering}
			\includegraphics[scale = 0.6]{Imagens/Inputs.png}
			\caption{Algorithm inputs \cite{augspurger2016experimental}}
			\label{tab:inputs}
		\end{table}

		The heat capacities of tool and workpiece materials are used as an interpolation function on the code, using data provided on tables \ref{tab:dataW} and \ref{tab:dataT}. The functions are given by the following equations:

		\begin{equation} 
		\label{eq_heatCapTool}
			c_{p}^{T} = 2.51\times 10^{- 10}\times T^{3} - 1.99\times 10^{- 6} \times T^{2} + 0.0027 \times T + 3.09
		\end{equation}

		\begin{equation} 
		\label{eq_heatCapWork}
			c_{p}^{W} = -4.39\times 10^{- 7}\times T^{3} - 7.07\times 10^{- 4} \times T^{2} + 0.0489 \times T + 481.21
		\end{equation}

		\begin{table}[h]
			\centering
			\captionsetup{justification=centering}
			\includegraphics[scale = 0.6]{Imagens/dataWorkpiece.png}
			\caption{Workpiece material data \cite{augspurger2016experimental}}
			\label{tab:dataW}
		\end{table}

		\begin{table}[h]
			\centering
			\captionsetup{justification=centering}
			\includegraphics[scale = 0.6]{Imagens/dataTool.png}
			\caption{Tool material data \cite{augspurger2016experimental}}
			\label{tab:dataT}
		\end{table}

	\section{Methods}
	\label{methods}
	
		\subsection{Power calculation}
			It is assumed that all mechanical work produced is used to generate heat in the primary shear zone \cite{abukhshim2006heat}. Hence, given the cutting force on the movement direction and the cutting velocity, the overall power produced and converted into heat is stated on equation \ref{eq_power}.

			\begin{equation} 
			\label{eq_power}
				P = F_{c}v_{c}
			\end{equation}
		
		\subsection{Thermal enegy - chip and tool}
			The methods used in this paper to calculate the heat flow through the tool and the energy carried away by chip are based on \cite{boothroyd1963temperatures}.

			\begin{figure}[H]
				\centering
				\captionsetup{justification=centering}
				\includegraphics[scale=0.6]{Cap4/ToolHeat.png}
				\caption{Heat flow through tool}
				\label{fig:heattool}
			\end{figure}

			Besides the temperature matrix, to calculate heat flow through the tool the following parameters are needed: the heat conductivity, the length of the chosen isothermal line, the temperature gradient normal to this isotherm and the width of the tool. The calculation is given by the following equation:

			\begin{equation} 
			\label{eq_heattool}
				\dot{Q}_{T} = kL\frac{dT}{dz}w
			\end{equation}

			For the energy carried away by the chip when it is flowing through control volume, the variables necessary to calculate this value are the heat capacity function $c_{p}(T)$ of workpiece, the chip temperature distribution along the line where the chip loses contact with tool $T_{C}^{out}$, the environment temperature $T_{e}$, the velocity of the chip normal to the line of end of contact $v_{chip}$, the chip thickness $t_{C}$ and the chip width $w$.

			\begin{figure}[H]
				\centering
				\captionsetup{justification=centering}
				\includegraphics[scale=0.6]{Cap4/energyChip.png}
				\caption{Thermal energy carried away by chip}
				\label{fig:energychip}
			\end{figure}

			The equation for this energy is represented below:

			\begin{equation} 
			\label{eq_energychip}
				\dot{Q}_{C}^{out} = c_{p}^{W}(T_{C}^{out} - T_{e})v_{chip}t_{C}w
			\end{equation}

		In this way, having the location and the temperature of each pixel related to the isotherms and the line of end of chip-tool contact, the math necessary to perform these equations is simple, providing reliable outcomes.

	\subsection{Volume control}

		For a matter of validation of the presented method and the lack of measurable temperatures on the workpiece surface, the control volume on figure \ref{fig:volControl} was designed.

		\begin{figure}[H]
			\centering
			\captionsetup{justification=centering}
			\includegraphics[scale=0.6]{Imagens/volumeControl.png}
			\caption{Control volume}
			\label{fig:volControl}
		\end{figure}

		The shear energy used to raise the temperature of the heat zones is calculated by means of equation \ref{eq_shear}

		\begin{equation} 
		\label{eq_shear}
		\dot{Q}_{shear} = F_{c}v_{c} - F_{p}v_{chip}
		\end{equation}

		It is estimated that $90\%$ of this energy generated in the primary shear zone ($\dot{Q}_{shear}$) is converted into sensible heat \cite{trigger1942}. The others $10\%$ are soon dissipated out the control volume. Thus, the energy balance of the control volume will provide:

		\begin{equation} 
		\label{eq_energybalance}
		\dot{Q}_{W} = P - \dot{Q}_{T} - \dot{Q}_{C}^{out} - 0.1\dot{Q}_{shear}
		\end{equation}
