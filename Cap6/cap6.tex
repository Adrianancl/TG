The results found when processing thermal images provided a reasonable understanding about heat distribution through tool and chip components. Most of the heat generated during the cutting process goes to dissipation on the removed chip, about 70\% of the total power generated. Most part of the data provided by the cutting process regards to transient state, but is also possible to note it reaching the steady state close to the end of the cutting process, which suggests that this computational method also may be extended for this part of the process.

One of the problems to elaborate this work was that many of the videos were damaged due to pieces of chip interfering on the ideal presentation of each thermal frame. Pieces of chip with different temperatures were captured on tool surface, disturbing the field of temperature along the tool shape. This fact made impossible the use of some frames from the same video and entire other experiments sometimes.

It may be noticed that for experiments with the same relation $v_{c}\times a_{p}$ (figure \ref{fig:energyChip}) seems to provide the same energy to be carried by the flowing chip. Since the thermography method is very sensible to external interference and many experiments were damaged as mentioned before, it would be necessary to perform new experiments to validate this hypothesis.

The thermography method for temperatures measurement still presents some challenges, mainly when comes to set the correct emissivity. Although it still takes a reasonable effort to determine the right emissivity for accomplishing a reliable measurement, the termography is a powerful tool for inspection, specially for cutting processes as discussed in this paper. With a filter of camera capable of measure temperatures lower than 200 Celsius degrees, it would be possible to complete the study with the measurement of temperatures on the workpiece area, providing more results.

Computer vision, as image recognition patterns and image processing, is being used each time more nowadays processes. For a future study beyond this paper, computer vision can become an even stronger tool when combined with machine learning, which is revolutionizing manufacturing and medical areas for example. The principles used to build this computational method could be converted to analyze others types of cutting processes, as milling. Then, it could be turned into an intelligent system to support machining processes, improving all cutting parameters in order to obtain higher efficiency of tool, increasing tool life, better surface finishing of the workpiece and lower cutting time.