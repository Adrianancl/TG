The results found when processing thermal images provided a reasonable understanding about heat distribution though tool and chip components. Most of heat generated during the cutting process goes to dissipation on the removed chip, about 70\% of the total power generated. All the data provided by the cutting process regards to transient state, but is also possible note it reaching the steady state close to the end of the cutting process, which suggests that this computacional method also may be extended for this process.

Many of the videos were damaged due to pieces of chip interfering on the ideal presentation of each thermal frame. With different temperature, the piece of chip were captured on tool surface which has disturbed the field of temperature along the tool shape. This fact made impossible the use of some frames from the same video and sometimes entire other experiments.

The thermography method for temperatures measurement still presents some challenges, mainly when comes to set the correct emissivity. Although it still takes a reasonable effort to determine the right emissivity for accomplishing a reliable measurement, the termography is a powerful tool for inspection when it is settled correctly , specially cutting processes as approached in this paper. With a filter of camera capable of measure temperatures lower than 200 Celsius degrees, it would be possible to complete the study with the measurement of temperatures on the workpiece area, providing more results.

Computer vision as image recognition patterns and image processing, for example, is being used each time more in nowadays processes. For a future study beyond this paper, computer vision can become a even stronger tool when combined with machine learning, which is revolutionizing manufacturing and medical areas. The principles used to build this computational method could be easily converted to analyze others types of cutting processes, as milling. Then, it could be turned into a intelligent system to support machining processes, improving all cutting parameters in order to obtain higher efficiency of tool, increasing tool life, better surface finishing of the workpiece and lower cutting time, for instance.